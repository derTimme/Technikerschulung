\documentclass[11pt,a4paper]{article}
\usepackage[utf8]{inputenc}
\usepackage[german]{babel}
\usepackage[T1]{fontenc}
\usepackage{amsmath}
\usepackage{amsfonts}
\usepackage{amssymb}
\usepackage{lmodern}
\usepackage{hyperref}
\renewcommand{\thepage}{}
\begin{document}
\begin{center}
\underline{\textbf{\Large Techniker- und Musikerschulung}}
\end{center}
\paragraph{Für wen} Techniker bzw. Technikverantwortliche in Gemeinden aller Art und Musiker, die in Bands spielen (oder das vorhaben) und mit Technikern zu tun haben.

\noindent
Details zum Inhalt der Schulung finden sich weiter unten.

%Die erste Einheit richtet sich an alle, die in Gemeinden mit Veranstaltungstechnik zu tun haben -- wir werden uns dabei auf Tontechnik beschränken, wenn besonderes Interesse an einer Schulung zu Lichttechnik besteht, kommt auf uns zu. Die zweite Einheit konzentriert sich dann auf den Umgang 

\paragraph{Wann und wo} Im Paul-Schneider-Haus (Paulinenstraße 15) in Reichenbach/Fils (\href{https://www.google.de/maps/place/Paul-Schneider-Haus/}{in Google Maps anschauen}) am Samstag, den 30.05.2015. Wir planen, morgens um 09:00 zu beginnen und werden die erste Einheit mit einem gemeinsamen Mittagessen um ca. 13:00 beenden. Die zweite Einheit ist dann ab dem Mittagessen bis ''Open End'', d. h. so lange Interesse (und Kaffee) da ist.

\paragraph{Kosten} Die Schulung an sich ist kostenlos, nur für das Mittagessen und evtl. Kaffee sollten ein paar Euro mitgebracht werden.

\paragraph{Inhalt} \textbf{TODO}

\paragraph{Mitbringen} \textbf{TODO}


\end{document}